\documentclass{masterthesis}

\usepackage{hyperref} % links
\usepackage{graphicx}
\usepackage{amsmath}
\usepackage{amssymb}
\usepackage{qcircuit}
\usepackage{braket}
\usepackage{float}
\usepackage[toc,page]{appendix}

\begin{document}

\title{Mathematical Modeling of Quantum Repeaters Chains}

\author{Lorenzo La Corte}

\advisor{}

\examiner{}

\maketitle

\chapter*{Mathematical Background}

\section*{Convolution}

Convolution is a fundamental mathematical operation used to combine two functions to produce a third function, which represents how the shape of one function is modified by the other. 

For two discrete functions \(a\) and \(b\), the convolution \(a * b\) is defined as:

\begin{equation}
    (a * b)(z) = \sum_{x=0}^{z} a(x) \cdot b(z - x)
\end{equation}

\paragraph*{Convolution of two Probability Distributions}
In the context of probability distributions, if \(X\) and \(Y\) are independent random variables with probability distribution functions \(p_X\) and \(p_Y\), their sum \(Z = X + Y\) has a probability distribution function \(p_Z\) given by the convolution of \(p_X\) and \(p_Y\):
\begin{equation}\label{eq:convolution}
    p_Z(z) = \sum_{x=0}^{z} p_X(x) \cdot p_Y(z - x)
\end{equation}

Thus, convolution is used to determine the probability distribution of the sum of independent random variables by combining their individual probability distributions.

\paragraph*{Example} 

Consider this simple example. Let $X$ and $Y$ discrete independent random variables. 

The probability distribution $\Pr(Z = z)$ of the random variable $Z = X + Y$ is computed as follows:
\begin{equation}
    \begin{array}{|c|c|c|c|c|c|c|}
        \hline
        x & \Pr(X = x) & y & \Pr(Y = y) & z & \Pr(Z = z) & \text{Derivation of }\Pr(Z = z)\\
        \hline
        0 & 0.2 & 0 & 0.3 & 0 & 0.06 & 0.2 \cdot 0.3 \\
        \hline
        1 & 0.5 & 1 & 0.4 & 1 & 0.23 & 0.2 \cdot 0.4 + 0.5 \cdot 0.3 \\
        \hline
        2 & 0.3 & 2 & 0.3 & 2 & 0.35 & 0.2 \cdot 0.3 + 0.5 \cdot 0.4 + 0.3 \cdot 0.3 \\
        \hline
         &  &  &  & 3 & 0.27 & 0.5 \cdot 0.3 + 0.3 \cdot 0.4 \\
        \hline
         &  &  &  & 4 & 0.09 & 0.3 \cdot 0.3 \\
        \hline
    \end{array}
\end{equation}

Note that $\Pr(Z = z)$ has five elements (all the possible sums), and it is valid as its probabilities sum to 1.

\begin{samepage}\label{page:convolution_associativity}
    The convolution operator \( * \) is associative, meaning that for any three functions \(a\), \(b\), and \(c\):
    \begin{equation}
        (a * b) * c = a * (b * c)
    \end{equation}        
\end{samepage}

\section*{Random Variables}

In this section, we fix notation on random variables and operations on them. 

Most random variables in the context of quantum repeaters
\begin{itemize}
    \item are discrete,
    \item have as domain a subset of nonnegative integers.
\end{itemize}

\paragraph*{PDF}\label{paragraph:pdf}
Let $X$ be such a random variable, then its probability distribution function is a map
\begin{equation}
    p_X : x \mapsto \Pr(X = x)
\end{equation} 
which describes the probability that its outcome will be $x \in \{0, 1, 2, \ldots \}$.

\paragraph*{CDF}\label{paragraph:cdf}
Equivalently, $X$ is described by its cumulative distribution function
\begin{equation}
    \Pr(X \leq x) = \sum_{y=0}^{x} \Pr(X = y),
\end{equation}

which is transformed to the probability distribution function as 
\begin{equation}
    \Pr(X = x) = \Pr(X \leq x) - \Pr(X \leq x - 1).
\end{equation}

\paragraph*{Independent Random Variables}\label{paragraph:independent_random_variables}
Two random variables $X$ and $Y$ are independent if 
\begin{equation}
    \Pr(X = x \text{ and } Y = y) = \Pr(X = x) \cdot \Pr(Y = y)
\end{equation}
for all $x$ and $y$ in the domain.

\paragraph*{Copies of a Random Variable}\label{paragraph:copies_of_a_random_variable}
By a \textit{copy} of $X$, we mean a fresh random variable which is independent from $X$ and identically distributed (i.i.d.).
We will denote a copy by a superscript in parentheses. For example, $X^{(1)}$, $X^{(142)}$ and $X^{(A)}$ are all copies of $X$.

The mean of $X$ is denoted by 
\begin{equation}\label{eq:expectation}
    E[X] = \sum_{x=0}^{\infty} \Pr(X = x) \cdot x
\end{equation}

and can equivalently be computed as 
\begin{equation}
    E[X] = \sum_{x=1}^{\infty} \Pr(X \geq x).
\end{equation}

\paragraph*{Function of Random Variables}\label{paragraph:function_of_random_variables}
If $f$ is a function which takes two nonnegative integers as input, then the random variable $f(X, Y)$ has probability distribution function defined as

\begin{equation}
    \Pr(f(X, Y) = z) := \sum_{\substack{x=0, y=0 \\ f(x,y)=z}}^{\infty} \Pr(X = x \text{ and } Y = y).
\end{equation}

\paragraph*{Sum of Random Variables}\label{paragraph:sum_of_random_variables}
An example of such a function is addition. 

Define $Z := X+Y$ where $X$ and $Y$ are independent, then the probability distribution $p_Z$ of $Z$ is given by 
\begin{equation}
    p_Z(z) = \Pr(Z = z) = \sum_{\substack{x=0, y=0 \\ x+y=z}}^{\infty} \Pr(X = x \text{ and } Y = y).
\end{equation}

But since $y = z - x$ this is equivalent to
\begin{align}
    p_Z(z) = \Pr(Z = z) &= \sum_{\substack{x=0}}^{z} \Pr(X = x \text{ and } Y = z - x) \\ 
                        &= \sum_{\substack{x=0}}^{z} \Pr(X = x) \cdot Pr(Y = z - x) \\
                        &= \sum_{x=0}^{z} p_X(x) \cdot p_Y (z - x)
\end{align}
which is the convolution of the distributions $p_X$ and $p_Y$, denoted as $p_Z = p_X * p_Y$ \hyperref[eq:convolution]{(see convolution)}.

Since convolution operator $*$ is associative, writing $a * b * c$ is well-defined, for functions $a$, $b$, $c$ from the nonnegative integers to the real numbers \hyperref[page:convolution_associativity]{(see associativity of convolution)}.
In general, \textbf{the probability distribution of sums of independent random variables equals the convolutions of their individual probability distribution functions}.

\section*{Geometric Distribution}\label{section:geometric_distribution}

The Geometric Distribution is a discrete probability distribution that models the number of trials needed to achieve the first success in a sequence of independent Bernoulli trials, each with the same success probability \( p \).

\subsection*{Probability Distribution Function (PDF)}

The Probability Distribution Function (PDF) of a Geometric Distribution gives the probability that the first success occurs on the \( t \)-th trial. It is defined as:
\begin{equation}
    \Pr(T = t) = p (1 - p)^{t-1} \quad \text{for} \quad t \in \{1, 2, 3, \ldots \},
\end{equation}
where:
- \( T \) is the random variable representing the trial number of the first success,
- \( p \) is the probability of success on each trial,
- \( (1 - p) \) is the probability of failure on each trial.

This formula expresses that the first \( t-1 \) trials must be failures (each occurring with probability \( 1 - p \)), and the \( t \)-th trial must be a success (with probability \( p \)).

\subsection*{Cumulative Distribution Function (CDF)}\label{subsection:geometric_cdf}

The Cumulative Distribution Function (CDF) of a Geometric Distribution gives the probability that the first success occurs on or before the \( t \)-th trial. It is defined as:
\begin{equation}
    \Pr(T \leq t) = 1 - (1 - p)^t.
\end{equation}

\paragraph*{Derivation of the CDF}

This is the derivation of the CDF of a Geometric Distribution, from its PDF
\begin{align}
    \Pr(T \leq t) &= 1 - \Pr(T > t) \\
    &= 1 - \sum_{k=t+1} \Pr(T = k) \\
    &= 1 - \left\{p (1 - p)^t + p (1 - p)^{t+1} + p (1 - p)^{t+2} + \ldots\right\} \\
    &= 1 - p (1 - p)^t \sum_{k=0} (1 - p)^k \\
    &= 1 - (1 - p)^t \sum_{k=0} p (1 - p)^k \\
    &= 1 - (1 - p)^t.
\end{align}

This CDF formula indicates the probability that the first success occurs within the first \( t \) trials.

\chapter*{Mathematical Model for \\ Waiting Time and Fidelity}

We derive expressions for the waiting time and fidelity of the first generated end-to-end link in the repeater chain protocol. 

We derive a recursive definition for the random variable $T_n$, which \textbf{represents the waiting time in a $2n$-segment repeater chain}.

Extending this definition to the Werner parameter $W_n$ of the pair, which stands in one-to-one correspondence to its fidelity $F_n$ using the equation:
\begin{equation}
    F_n = \frac{1 + 3 W_n}{4}.
\end{equation}

Note that all operations in the repeater chain protocols we study
\begin{itemize}
    \item Entanglement generation over a single hop
    \item Distillation
    \item Swapping
\end{itemize}
take a duration that is a multiple of ${L_0}/{c}$, the time to send information over a single segment.

For this reason, it is common to denote the waiting time in \textbf{discrete units} of ${L_0}/{c}$, which is a convention we comply with for $T_n$.

Regarding cutoffs, ... % TODO

\section*{Heraldeld Entanglement Generation}

\subsection*{Waiting Time for Elementary Entanglement}
In modeling the random variable $T_n$, which represents the waiting time in a $2^n$ segment repeater chain, we can reason by induction.

The base case \textbf{$T_0$ is the waiting time for the generation of elementary entanglement}.

Since we model the generation of single-hop entanglement by attempts which succeed with a fixed probability $p_{\text{gen}}$, the waiting time $T_0$ is a discrete random variable (in units of $L_0 /c$) which follows a \hyperref[section:geometric_distribution]{geometric distribution} with probability distribution given by 
\begin{equation}
    \Pr(T_0 = t) = p_{\text{gen}} (1 - p_{\text{gen}})^{t-1} \quad \text{for} \quad t \in \{1, 2, 3, \ldots \}.
\end{equation}

For what follows, it will be more convenient to specify $T_0$ by its \hyperref[subsection:geometric_cdf]{cumulative distribution function} (CDF), which is given by
\begin{equation}
    \Pr(T_0 \leq t) = 1 - (1 - p_{\text{gen}})^t.
\end{equation}

\subsection*{Werner Parameter for Elementary Entanglement}
... % TODO

\section*{Entanglement Swapping}

Once we have generated elementary entanglement, we can use it to create entanglement over longer distances by entanglement swapping.

We defined $T_0$ as the waiting time for the generation of elementary entanglement, and our base for the induction.

We now define our inductive step assuming that we have found an expression for $T_n$ and we want to construct $T_{n+1}$. 

In order to perform the entanglement swap to produce a single $(2^n+1)$-hop link, a node needs to wait for the production of two $(2^n)$-hop links, one on each side. 

Denote the waiting time for the pairs by $T_n^{(A)}$ and $T_n^{(B)}$, both of which are i.i.d. with $T_n$, as they are \hyperref[paragraph:copies_of_a_random_variable]{copies} of it. 

The time until both pairs are available is now given by
\begin{equation}
    M_n := \max(T_n^{(A)} , T_n^{(B)})
\end{equation}

which is distributed according to
\begin{equation}
    \Pr(M_n \leq t) = \Pr(T_n^{(A)} \leq t \text{ and } T_n^{(B)} \leq t) = \Pr(T_n \leq t) ^ 2.
\end{equation}

\end{document}