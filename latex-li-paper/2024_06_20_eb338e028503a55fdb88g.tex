\documentclass[10pt]{article}
\usepackage[utf8]{inputenc}
\usepackage[T1]{fontenc}
\usepackage{amsmath}
\usepackage{amsfonts}
\usepackage{amssymb}
\usepackage[version=4]{mhchem}
\usepackage{stmaryrd}

\begin{document}
attempts. We can write its waiting time $M$ as

$M=\sum_{k}\left\{\left[Y_{\text {cut }}^{(k)} \prod_{j=1}^{k-1}\left(1-Y_{\text {cut }}^{(j)}\right)\right] \cdot\left[Z_{\mathrm{s}}^{(k)}+\sum_{i=1}^{k-1}\left(Z_{\mathrm{f}}^{(i)}\right)\right]\right\}$.

This expression will replace $M=\max \left(T_{\mathrm{A}}, T_{\mathrm{B}}\right)$ used in (4). For $\tau=\infty$ or $w_{\text {cut }}=0$, i.e., no cutoff, $Y_{\text {cut }}$ is always 1 . Therefore, $k=1$ is the only surviving term and the two expressions coincide.

To calculate the waiting time distribution, we need three joint distributions- $P_{\mathrm{f}}^{\prime}$ for unsuccessful input link preparation because of the cutoff, $P_{\mathrm{s}, \mathrm{f}}^{\prime}$ for successful preparation but unsuccessful swap/distillation, and $P_{\mathrm{s}, \mathrm{S}}^{\prime}$ for both successful

$$
\begin{aligned}
P_{\mathrm{f}}^{\prime}(t)= & \operatorname{Pr}\left(M=t, Y_{\mathrm{cut}}=0\right) \\
= & \sum_{t_{\mathrm{A}}, t_{\mathrm{B}}: t_{\mathrm{fail}}\left(t_{\mathrm{A}}, t_{\mathrm{B}}\right)+t_{\mathrm{c}}=t} \operatorname{Pr}\left(T_{\mathrm{A}}=t_{\mathrm{A}}, T_{\mathrm{B}}=t_{\mathrm{B}}\right) \\
& \cdot\left[1-p_{\mathrm{cut}}\right]\left(T_{\mathrm{A}}, T_{\mathrm{B}}\right) \\
P_{\mathrm{s}, \mathrm{f}}^{\prime}(t)= & \operatorname{Pr}\left(M=t, Y_{\mathrm{cut}}=1, Y=0\right) \\
= & \sum_{t_{\mathrm{A}}, t_{\mathrm{B}}: \max \left(t_{\mathrm{A}}, t_{\mathrm{B}}\right)+t_{\mathrm{c}}=t} \operatorname{Pr}\left(T_{\mathrm{A}}=t_{\mathrm{A}}, T_{\mathrm{B}}=t_{\mathrm{B}}\right) \\
& \cdot\left[p_{\mathrm{cut}} \cdot(1-p)\right]\left(t_{\mathrm{A}}, t_{\mathrm{B}}\right) \\
P_{\mathrm{s}, \mathrm{S}}^{\prime}(t)= & \operatorname{Pr}\left(M=t, Y_{\mathrm{cut}}=1, Y=1\right) \\
= & \sum_{t_{\mathrm{A}}, t_{\mathrm{B}}: \max \left(t_{\mathrm{A}}, t_{\mathrm{B}}\right)+t_{\mathrm{c}}=t} \operatorname{Pr}\left(T_{\mathrm{A}}=t_{\mathrm{A}}, T_{\mathrm{B}}=t_{\mathrm{B}}\right) \\
& \cdot\left[p_{\mathrm{cut}} \cdot p\right]\left(t_{\mathrm{A}}, t_{\mathrm{B}}\right)
\end{aligned}
$$

The prime notation indicates that they describe the waiting time of one attempt in CUTOFF, in contrast to one attempt in swap or distillation.

For one attempt in swap/distillation with time-out, we then get similarly to (7)

$$
\begin{aligned}
& P_{\mathrm{S}}(t)=\operatorname{Pr}(M=t, Y=1)=\sum_{k}\left[\left(\begin{array}{c}
k-1 \\
j=1
\end{array} P_{\mathrm{f}}^{\prime(j)}\right) * P_{\mathrm{s}, \mathrm{s}}^{\prime}\right](t) \\
& P_{\mathrm{f}}(t)=\operatorname{Pr}(M=t, Y=0)=\sum_{k}\left[\left( P_{\mathrm{f}}^{\prime(j)}\right) * P_{\mathrm{s}, \mathrm{f}}^{\prime}\right](t)
\end{aligned}
$$

as well as the expressions in Fourier space analogous to

$$
\begin{aligned}
& P_{\mathrm{s}}(t)=\operatorname{Pr}(M=t, Y=1)=\mathcal{F}^{-1}\left[\frac{\mathcal{F}\left[P_{\mathrm{s}, \mathrm{s}}^{\prime}\right]}{1-\mathcal{F}\left[P_{\mathrm{f}}^{\prime}\right]}\right] \\
& P_{\mathrm{f}}(t)=\operatorname{Pr}(M=t, Y=0)=\mathcal{F}^{-1}\left[\frac{\mathcal{F}\left[P_{\mathrm{s}, \mathrm{f}}^{\prime}\right]}{1-\mathcal{F}\left[P_{\mathrm{f}}^{\prime}\right]}\right]
\end{aligned}
$$

The total waiting time then follows by substituting the expressions for $P_{\mathrm{f}}$ and $P_{\mathrm{S}}$ above in (7) or (11).

For entanglement swap, i.e., constant success probability $p_{\text {swap, }}$, simplification can be made for this calculation. In this special case, $P_{\mathrm{s}, \mathrm{f}}^{\prime}$ and $P_{\mathrm{s}, \mathrm{S}}^{\prime}$ differ only by a constant and the same holds for $P_{\mathrm{s}}$ and $P_{\mathrm{f}}$.

\section*{2) WERNER PARAMETER}
For the Werner parameter, we now need three steps.

We start from calculating the resulting Werner parameter of a swap or distillation for the very last preparation attempt where $Y_{\text {cut }}=Y=1$. It is denoted by $W_{\mathrm{s}}^{\prime}$ and we only need to replace $P_{\mathrm{s}}$ by $P_{\mathrm{s}, \mathrm{S}}^{\prime}$ and $p \cdot w_{\text {out }}$ by $p_{\mathrm{cut}} \cdot p \cdot w_{\text {out }}$ in (12).

Next, we compute the Werner parameter $W_{\mathrm{s}}(t)$ as a function of time $t$ that includes the failed cutoff attempts, in ana$\log$ to the derivation of (13). $W_{\mathrm{s}}(t)$ is the Werner parameter that the pair of output links of cutofF will produce, given that the swap or distillation operation following is successful:

$$
W_{\mathrm{s}}(t)=\frac{\sum_{k=1}^{\infty}\left[\left(\begin{array}{c}
k-1 \\
* \\
j=1
\end{array} P_{\mathrm{f}}^{\prime}\right) *\left(P_{\mathrm{s}, \mathrm{s}}^{\prime} \cdot W_{\mathrm{s}}^{\prime}\right)\right](t)}{P_{\mathrm{s}}(t)}
$$

Finally, we consider the time consumed by failed attempts in SWAP or DIST and obtain

$$
W_{\text {out }}(t)=\frac{\sum_{k=1}^{\infty}\left[\binom{k-1}{\multirow{j=1}{*}{P_{\mathrm{f}}}} *\left(P_{\mathrm{S}} \cdot W_{\mathrm{s}}\right)\right](t)}{\operatorname{Pr}\left(T_{\text {out }}=t\right)}
$$

Using the Fourier transform, the two expressions above become

$$
\begin{aligned}
W_{\mathrm{s}}(t) & =\mathcal{F}^{-1}\left[\frac{\mathcal{F}\left[P_{\mathrm{s}, \mathrm{s}}^{\prime} \cdot W_{\mathrm{s}}^{\prime}\right]}{1-\mathcal{F}\left[P_{\mathrm{f}}^{\prime}\right]}\right] \frac{1}{P_{\mathrm{s}}} \\
W_{\text {out }}(t) & =\mathcal{F}^{-1}\left[\frac{\mathcal{F}\left[P_{\mathrm{s}} \cdot W_{\mathrm{s}}\right]}{1-\mathcal{F}\left[P_{\mathrm{f}}\right]}\right] \frac{1}{\operatorname{Pr}\left(T_{\text {out }}=t\right)}
\end{aligned}
$$

\section*{F. CONVERTING THE CLOSED-FORM EXPRESSIONS INTO AN EFFICIENT ALGORITHM}
In the sections above, we presented closed-form expressions for $T_{\text {out }}$ and $W_{\text {out }}$ for each of the four PROTOCOL-UNITS, as a function of waiting time distribution and Werner parameter of the input links. In order to convert these expressions into an algorithm, we take the same approach as in [18] and cap the infinite sum in (7) and (13) by a prespecified truncation time $t_{\text {trunc }}$. This yields a correct $\operatorname{Pr}\left(T_{\text {out }}=t\right)$ and $W_{\text {out }}(t)$ for $t \in\left\{1, \ldots, t_{\text {trunc }}\right\}$ since in each of the expressions with an infinite sum above, $\operatorname{Pr}\left(T_{\text {out }}=t\right)$ and $W_{\text {out }}(t)$ are only dependent on waiting time and Werner parameter of input links produced at time $t^{\prime} \leq t$.

We now show that the algorithm scales polynomially in terms of $t_{\text {trunc. }}$. To analyze the complexity, we divide the algorithm into two parts: computing the distribution for one attempt, i.e., the iteration over all possible values of $T_{\mathrm{A}}, T_{\mathrm{B}}$ [(5), (6), and (12)] and for the whole PROTOCOL-UNIT[(7) and (13)].

The complexity for the first part is $\mathcal{O}\left(t_{\text {trunc }}^{2}\right)$ since it iterates over two discrete random variables up to $t_{\text {trunc }}$. For the second part, because we need at least one time step in each attempt, i.e., $\operatorname{Pr}(T=0)=0$, only the first $t_{\text {trunc }}$ convolutions will have nonzero contribution. We can perform the convolution iteratively for each $k$ using at most $t_{\text {trunc }}$ convolutions. The complexity of one convolution with fast Fourier transform (FFT) is $\mathcal{O}\left(t_{\text {trunc }} \log t_{\text {trunc }}\right)$ [28]. Thus, the complexity of the


\end{document}